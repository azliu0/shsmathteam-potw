The key idea for this problem is that while it \textit{can} be solved with algebra, it's better solved using a geometric analogy. For completeness, we'll present both the intended geometric solution, and an alternate algebraic solution.\V

\begin{solution}
\textbf{Solution 1} (intended)\textbf{:}\V

\begin{center}
\begin{asy}
import graph; import math; size(6cm); 
real labelscalefactor = 1.2; 
pen dps = linewidth(0.7) + fontsize(10); defaultpen(dps); 
pen dotstyle = black+3; 
real xmin = -2, xmax = 12, ymin = -2, ymax = 8;  
pen wwwwww = rgb(0.4,0.4,0.4); 

path r = unitcircle;
draw(r);

pair A = dir(90);
pair B = dir(210);
pair C = dir(270);
pair D = dir(10);

draw(A--B--C--D--cycle, blue);
draw(A--C, dotted);
draw(B--D, dotted);

path rightanglemark(pair A, pair B, pair C, real s=2.75)
{ 
	pair P,Q,R; 
	P=s*0.03*unit(A-B)+B; 
	R=s*0.03*unit(C-B)+B; 
	Q=P+R-B; 
	return P--Q--R;
}
draw(rightanglemark(A,B,C));
draw(rightanglemark(A,D,C));
draw(arc(A, 0.15, 240, 320), black);
label("$\varphi$", A, dir(280)*4.5, black); 

dot(A,dotstyle); 
label("$A$", A, NW); 
dot(B,dotstyle); 
label("$B$", B, W); 
dot(C,dotstyle); 
label("$C$", C, SE); 
dot(D,dotstyle); 
label("$D$", D, E); 
label("$a$", A--B, blue);
label("$b$", B--C + (0.03,0.03), blue);
label("$c$", C--D, blue);
label("$d$", D--A, blue);
\end{asy}
\end{center}

Let $a$, $b$, $c$, and $d$ be the side lengths of cyclic quadrilateral $ABCD$ (for now, we don't actually know that this quadrilateral exists, but we'll proceed under this presumption and cross our fingers that the equality cases work out). \V

Let $\theta = \angle{ABC} = 180^{\circ} - \angle{ADC}$, and $\varphi = \angle{BAD} = 180^{\circ}-\angle{DCB}$. Now, from the law of cosines, we have:
\begin{align*}
    AC^2 &= a^2+b^2 - 2ab\cos{\theta} = c^2+d^2+2cd\cos{\theta} \tag{1}\\
    BD^2 &= a^2+d^2 - 2ad\cos{\varphi} = b^2+c^2+2bc\cos{\varphi}.\tag{2}
\end{align*}

From equation 1, we have $0 = a^2+b^2-c^2-d^2 = 2\cos{\theta}(ab+cd)$, therefore $\theta = 90^{\circ}$. Moreover, from equation 2 we get that $\frac{56}{53}(bc+ad) = a^2-b^2-c^2+d^2 = 2\cos{\varphi}(ad+bc)$, so $\cos{\varphi} = \frac{28}{53}$ and $\sin{\varphi} = \frac{45}{53}$. Now, we have that:
\[[ABCD] = \frac{1}{2}(ab+cd) = \frac{1}{2}\sin{\varphi}(ad+bc),\]
giving us a final answer of 
\[\frac{ab+cd}{bc+ad} = \frac{45}{53}.\]
Note also that equality cases should be easy to construct based on $\theta$ and $\varphi$, so we are done.
\end{solution}\newpage

\begin{solution}
\textbf{Solution 2} (algebraic, by tastymath)\textbf{:}\V

Adding and subtracting our two conditions gives us:
\begin{align*}
    a^2-c^2 &= \frac{28}{53}(bc+ad) \\
    b^2-d^2 &= -\frac{28}{53}(bc+ad).
\end{align*}
We also have the identity $(a^2-c^2)(b^2-d^2) = (ac+bd)^2-(ad+bc)^2$ (this can be motivated by norms in the complex plane, i.e, $\lVert a+bi\rVert\lVert c+di\rVert = \lVert (a+bi)(c+di)\rVert$). Thus, if we multiply together the above two equations, we get that 
\[(ac+bd)^2-(bc+ad)^2 = -\left(\frac{28}{53}\right)^2(bc+ad)^2,\]
implying that 
\[\frac{ab+cd}{bc+ad} = \frac{45}{53},\]
as desired.

\end{solution}