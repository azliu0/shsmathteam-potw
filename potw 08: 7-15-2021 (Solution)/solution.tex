% potw#5_6/24/2021_solution.pdf
For this problem, the key is to observe the miraculous identity:
\[a^3+b^3+c^3-3abc = (a+b+c)(a^2+b^2+c^2-ab-ac-bc).\]
The rest of the problem boils down to rote algebra.\V

\begin{solution}
\textbf{Solution}\textbf{:}\V

Rewrite the given equation as 
\[x^3+y^3+(-1)^3 - 3(x)(y)(-1) = 0.\]
Then, we can factor the left hand side so that 
\[(x+y-1)(x^2+y^2+1-xy+x+y) = 0.\]
This implies that $x+y=1$, or $x^2+y^2+1-xy+x+y=0$. To deal with the latter equation, we can proceed in one of two ways: 

\vspace{0.15cm} \rule{\textwidth}{0.3pt}\vspace{0.3cm}

\textit{Method 1:} Treat the equation as a quadratic in $x$: 
\[x^2 + (1-y)x + y^2+y+1 = 0.\] 
Taking the discriminant for real solutions, we see that we must have 
\begin{align*}
    (y-1)^2 - 4(y^2+y+1)&\geq 0 \\
    \iff (y+1)^2\leq 0,
\end{align*}
so $x=y=-1$ is our only solution. 

\vspace{0.15cm} \rule{\textwidth}{0.3pt}\vspace{0.3cm}

\textit{Method 2:} Multiplying both sides of the equation by $2$ and rearranging squared terms, we see that it factors as
\[(x-y)^2+(x+1)^2+(y+1)^2=0,\]
so again our only valid solution is $x=y=-1$.

\vspace{0.15cm} \rule{\textwidth}{0.3pt}\vspace{0.3cm}

With either method, we see that the solution set is the line $x+y=1$ and the point $(-1,-1)$. 
\begin{center}
\begin{asy}
import graph; size(3cm); 
real labelscalefactor = 0.5; 
pen dps = linewidth(0.7) + fontsize(8); defaultpen(dps); 
pen dotstyle = black;
real xmin = -3, xmax = 3, ymin = -2.778182196853986, ymax = 3.10759057416779;
real markscalefactor=0.02;
path rightanglemark(pair A, pair B, pair C, real s=8)
{ 
	pair P,Q,R; 
	P=s*markscalefactor*unit(A-B)+B; 
	R=s*markscalefactor*unit(C-B)+B; 
	Q=P+R-B; 
	return P--Q--R;
}

Label laxis; laxis.p = fontsize(10); 
xaxis(xmin, xmax, Arrows(2)); 
yaxis(ymin, ymax, Arrows(2)); 

real f(real x) { return 1-x;}
draw(graph(f,-2,3.0), red+linewidth(0.3), Arrows(3));
draw((-1,-1)--(0.5,0.5), red+dotted+linewidth(0.3));
draw(rightanglemark((-1,-1),(0.5,0.5),(1,0)), red+linewidth(0.3));

real s=sqrt(3)/2;

dot((-1,-1),2+red); 
dot((0.5,0.5),1+red); 
dot((0.5+s,0.5-s),2+red); 
dot((0.5-s,0.5+s),2+red); 
draw((-1,-1)--(0.5+s,0.5-s), red+linewidth(0.3));
draw((-1,-1)--(0.5-s,0.5+s), red+linewidth(0.3));
fill((-1,-1)--(0.5+s,0.5-s)--(0.5-s,0.5+s)--cycle, red+opacity(0.05));

label(scale(0.75)*"$(-1,-1)$", (-1,-1), S); 
label(scale(0.75)*"$(0.5,0.5)$", (0.5,0.5), NE); 
clip((xmin,ymin)--(xmin,ymax)--(xmax,ymax)--(xmax,ymin)--cycle); 
\end{asy}
\end{center}

The distance between the point and line is equal to $3\sqrt{2}/2$, which is also equal to the height of our equilateral triangle, so our desired area is 
\[\frac{\sqrt{3}}{4}\left(\sqrt{6}\right)^2 = \boxed{\frac{3}{2}\sqrt{3}}.\]

\end{solution}\V
