% potw#5_6/24/2021_solution.pdf

First, a brief exposition on finite differences: given any polynomial $f(x)$ with degree $d$, consider the sequence $f(a), f(a+1), f(a+2), \hdots$, for some arbitrary $a$. Then, we define the following sequences of finite differences: 
\begin{itemize}
    \item 1st sequence: $f(a+1)-f(a)$, $f(a+2)-f(a+1)$, $f(a+3)-f(a+2)$, $\hdots$ 
    \item 2nd sequence: $f(a+2)-2f(a+1)+f(a)$, $f(a+3)-2f(a+2)+f(a+1)$, $\hdots$
    \item $\hdots$ 
\end{itemize}
We create each new sequence by taking the difference between adjacent terms in the previous sequence.\V

For concreteness, consider the example $f(x) = x^2$ and $a=0$. Then, our original sequence reads $0, 1, 4, 9, 16, 25, 36, \hdots$, giving us the following sequences of finite differences: 
\begin{itemize}
    \item 1st sequence: $1, 3, 5, 7, 9, 11, \hdots$
    \item 2nd sequence: $2, 2, 2, 2, 2, \hdots$
    \item 3rd sequence: $0, 0, 0, 0, 0, \hdots$
\end{itemize}

Important properties of these sequences:
\begin{theorem}
\textbf{Theorem 1 (Finite Differences).} Let $f(x)$ be a polynomial with degree $d$ and leading coefficient $a_d$.  

\begin{enumerate}
    \item [(a).] The $i$-th term of the $n$-th sequence of finite differences is given by \[\sum_{k=0}^{n}(-1)^n\binom{n}{k}f(a+k+i)\]
    \item [(b).] All elements of the $d$-th sequence are equal to the constant $c$, where $c = a_d\cdot d!$. All elements of the $d+1$-th sequence are $0$. 
\end{enumerate}
\end{theorem}\V

Both parts of this theorem can be proven inductively. For part $(a)$, adjacent terms of the sequence will have coefficients for each $f($something$)$ that sum easily with pascal's identity. For part $(b)$, if we suppose that $f(x) = a_dx^d+\hdots$, then 
\[f(x+1)-f(x) = a_d(x+1)^d - a_d(x)^d + \hdots = a_d(d)x^{d-1}+\hdots,\]
etc. and we can continue inducting downwards. \V

We're now ready to present our solution. Both use finite differences; the first utilizes part $(b)$ and a manual computation of each sequence of finite differences, while the second uses a direct application of the formula in part $(a)$. \newpage

\begin{solution}
\textbf{Solution 1 (benq)}\textbf{:}\V

Consider the sequence $k, F_3, F_5, \hdots, F_{19}$, corresponding to $P(0), P(2), \hdots, P(18)$. Then: 
\begin{itemize}
    \item the 1st sequence of finite differences: $(F_3-k)$, $F_4$, $F_6$, $\hdots$, $F_{18}$ 
    \item the 2nd sequence of finite differences: $(F_4-F_3+k)$, $F_5$, $F_7$, $\hdots$, $F_{17}$
    \item the 3rd sequence of finite differences: $(F_5-F_4+F_3-k)$, $F_6$, $F_8$, $\hdots$, $F_{16}$
    \item $\vdots$
    \item the 8th sequence of finite differences: $(F_{10}-F_9+\hdots -F_3+k)$, $F_{11}$. 
\end{itemize}

By Theorem $1$, we know that the $9$th sequence is constant; therefore, the next term in the $8$th sequence is equal to $F_{11} + (F_{11}-F_{10}+F_9-\hdots +F_3-k)$. This allows us to calculate the next term in each subsequent sequence "higher-up" in the ladder, so that we get that 
\begin{align*}
    P(20)+k &= (F_{11}-F_{10}+F_9-\hdots F_3) + (F_{11}+F_{12}+\hdots F_{19}) \\
    &= (F_{10}+1) + (F_{21}-F_{12}) \\ 
    &= F_{21} - F_{11} + 1.
\end{align*}

Therefore, $k = \frac{F_{21}-F_{11}+1}{2} = 5429$.
\end{solution}\V


\begin{solution}
\textbf{Solution 2 (benq)}\textbf{:}\V

By Theorem 1, we have that 
\[0 = \sum_{k=0}^{10}(-1)^n\binom{n}{k}P(2k),\]
which rearranges to 
\[k+P(20) = \sum_{k=1}^{9}(-1)^{k+1}\binom{n}{k}F_{2k+1}.\]

Now ... messy algebra, details of which are left out here (:\newline
Using \href{https://artofproblemsolving.com/wiki/index.php/Binet\%27s_Formula}{Binet's formula} and the binomial theorem, the right hand side collapses to $F_{21}-F_{11}+1$, so done.
\end{solution}